\documentclass[12pt,a4paper]{article}

\usepackage[a4paper]{geometry}
\usepackage[utf8]{inputenc}
\usepackage[unicode,final,colorlinks=true]{hyperref}        

\begin{document}

\title{De goden zijn microben}
\author{Rowan Rodrik van der Molen}
\date{juni 2018}
\maketitle


Als puber vond ik “wetenschap ook maar een religie”. Dat was vrij irritant voor de wetenschappers in mijn persoonlijk omgeving die me zo nu en dan probeerden te wijzen op simpele, wetenschappelijk vastgestelde feiten. Als jongvolwassene culmineerde mijn wetenschaps-skepticisme zich zelfs in een creationistische fase. Via het Internet, werd ik bijna nog verleid tot een paranoia Christelijk samenzweringsfundamentalisme.  Mijn grootouders van mijn moeder's kant hadden beiden biologie gestudeerd. Ook mijn vader was grotendeels atheïstisch opgevoed, en mijn ouders waren allebei zeer liberaal in hun politieke oriëntatie. Er was geen betere manier om me af te zetten dan door met bijbelse onzin aan 

De vergissing was makkelijk gemaakt. Ik had geen zin om te luisteren naar de hooggeleerde heren—ja, het waren vooral heren—die me “de waarheid” aan mijn verstand probeerden te peuteren, en dacht: de waarheid, dat is toch iets van de kerk? Ik stelde me een universiteit voor als een soort kathedraal, hoofdzakelijk gericht op het in stand houden van autoriteitsstructuren. De waarheid leek me meer iets persoonlijks, subjectiefs. Ik was een postmodernist 20 jaar voordat ik voor het eerst van het woord gehoord had. Als MAVO scholier (tegenwoordig VMBO-TL) zat er ook niet veel in het schoolcurriculum dat mijn neus de andere kant op richtte.

Iedere wetenschapsfilosoof had me kunnen uitleggen dat de schoonheid van wetenschap juist is dat ik deze heren niet op hun blauwe ogen (of hun bombast) hoefde te geloven. Ongeacht hun en mijn persoonlijke gebreken, st

Maar de vraag is nu niet of wetenschap een religie is. De vraag is of er binnen de wetenschap ruimte is voor religie. Laat ik eerst eens te raad gaan bij mijn puberbrein. Mijn onvolgroeide, onzekere, overzelfverzekerde zelf zou moeiteloos het onderwerp gekaapt hebben met: “Dus wetenschap is een methode, geen religie; maar geloof in de wetenschap kan wel religieuze vormen aannemen.” En dan snap ik weer waarom men soms knarsetandend met mij aan tafel zat. Arme mannen. (Nou ja, ze waren best welvarend, zoals de meeste hoogopgeleide NRC lezers.)

Dawkins, een vokaal tegenstander van religie noemen het een “mislukte wetenschap”

Ik had niks met objectieve waarheid, want ik was niet erg geleerd en kon me als puber moeilijk omhoogvechten in de dominantiehierarchy op basis van mijn MAVO kennis. De waarheid was ook maar een mening, aangezien ik mijn mening tenminste kon verdedigen (op basis van virulente persoonlijk aanvallen).


Via een onweg kwam ik uiteindelijk toch in het wetenschappelijk onderwijs terecht.

De goden zijn microben.

Dat is mijn antwoord op de vraag of er binnen de wetenschap ruimte is voor religie. Ja. Niet alleen is er ruimte voor religie. Wetenschap heeft religie nodig. En religie heeft wetenschap nodig.

\end{document}
