\documentclass[12pt,a4paper]{article}

\usepackage[a4paper]{geometry}
\usepackage[utf8]{inputenc}
\usepackage{textcomp}
\usepackage[unicode,final,colorlinks=true]{hyperref}        

\begin{document}

\title{Het eerste gebod}
\author{Rowan Rodrik van der Molen}
\date{juni 2018}
\maketitle


% Domme MAVO puber, die liever PC-spellen speelde dan naar school ging.
% Pompeuze, wetenschapsgeile vaders.
% Argumenten vanuit autoriteit.
% Revolutionaire samenzweringsideeën.
% Verkeerd beeld van wetenschap.
% Inconsistente New Age zweverij.
% Scepsis richting wetenschap.
% Atheïstische identiteitsvorming.
% De goden zijn microben.
% God is niet grappig.
% Gij zult klein zijn.
% En anders wordt je wel klein gestampt.
% Wetenschaps-scepsis als kern van religie.
% Het onzichtbare in onszelf als geloof.
% Zingeving als beginpunt voor wetenschap.


De goden zijn microben – klein — zo klein dat ze onzichtbaar zijn. “Gij zult klein zijn,” is daarom een vanzelfsprekend eerste gebod voor micro-gelovigen. Een µ-gelovige word je vanzelf door door een microscoop te kijken en na te denken over wat je ziet.

De microscoop legt een wereld bloot die fundamenteler is dan de onze: de wereld van cellen. De meeste cellen zijn vrijlevend, maar sommige hebben zich in de loop van de evolutie in zulke complexe kolonies georganiseerd dat we die lichamen zien als organismen op zich. Als mens ben ik een gigantisch complex van cellen – allemaal met identiek DNA en per weefsel-type verschillende gen-expressie. Het is een duizelingwekkend idee: enkele tientallen biljoenen cellen die nauw genoeg kunnen samenwerken om een essay over religie en wetenschap te schrijven.

% Wat de prestaties van ons lichaam extra indrukwekkend maakt is het feit dat al onze lichaamscellen voortkomen uit delingen van één enkele cel — de zygote, die is ontstaan uit een forse geslachtscel van onze moeder en een veel kleinere geslachtscel van onze vader. En dit gaat zo terug, tot gemeenschappelijke voorouders die helemaal niet op onze vader òf moeder leken, en uiteindelijk tot een ongeslachtelijk, ééncellig organisme dat we alleen met hulp van de microscoop kunnen zien.

Microben – eencellige organismen – waren er al miljarden jaren voordat de eerste mens op het evolutionaire toneel verscheen. De mens is een tijdelijke vertoning in een soortenrijk waar meercelligen vooral bijrollen vervullen. Onze bijrol is bijzonder; we zijn er in geslaagd onze biosfeer om te vormen, maar zonder microben was er geen biosfeer, en nog lang nadat de resten van de laatste mens door microben zijn omgezet zal de biosfeer door microben gedomineerd worden.

Het menselijk beestje mag dan uit een indrukwekkend aantal cellen bestaan, al het dierlijk en plantaardig leven bij elkaar opgeteld stelt weinig voor tegenover de microbiële biomassa. Zelfs binnen ons eigen lichaam wordt het aantal microbiële cellen geschat op zo'n honderd biljoen, zo'n 3 tot 10 keer meer dan onze lichaamseigen cellen. In onze darmen zit ruim een kilo aan godengebroed.

Waarom noem ik die microben goden? Het lijkt misschien alsof ik een beetje grappig probeer te doen. Schijn bedriegt. Dit is een oprechte poging om je religieuze affiliatie te winnen. Microben bepalen in angstwekkende mate ons lot. Pas sinds de uitvinding van de microscoop (door Antonie van Leeuwenhoek in de zeventiende eeuw) heeft de wetenschap schoorvoetend middelen ontwikkeld om bacteriën te kunnen bestrijden. (En nog steeds is necrotiserende fasciitis geen lolletje.) En nu blijkt dat we die kilo microben in onze ingewanden juist nodig hebben. En microben spelen een essentiële rol in bodems en water. We kunnen de goden niet doden zonder onszelf te doden. Daarom willen wij µ-gelovigen ze beter begrijpen.

Om microben te begrijpen moet je je klein maken, heel klein – zo klein als ik me voelde als angstige puber in de laat-twintigste eeuw… Ik zat op de MAVO (tegenwoordig VMBO-TL) – niet bepaald het voorbereidend wetenschappelijk onderwijs. Mijn ouders hadden allebei het gymnasium gedaan en hadden elkaar ontmoet op de universiteit, waar mijn vader biologie studeerde en mijn opa (van moeders kant) biologie doceerde. Ik kwam dus uit een intellectuele dynastie, en, zoals in zoveel intellectuele nesten, werd er dringend neergekeken op Jan met de Pet (tegenwoordig Henk en Ingrid), terwijl ik, als ik even niet aan het gamen of Internetten was, in de schoolbanken zat te leren voor Henk. Dat was niet sjiek.

Op de MAVO werd nauwelijks aandacht besteed aan wetenschappelijke vorming. De docenten waren in principe al blij als ze ons enigszins konden leren begrijpelijk te schrijven en begrijpend te lezen. En we moesten natuurlijk een paar sommetjes leren. Op weg naar volwassenheid was mijn voornaamste aanraking met wetenschap via de hooggeleerde heren in mijn persoonlijke omgeving en via de TV. Van \emph{The X-Files} leerde ik dat agent Mulder—vol fantastisch bijgeloof—het vaker bij het rechte eind had dan agent Scully met haar hoofd vol wetenschappelijk scepticisme. Het leek voor mij als jonge kijker alsof zij, verblind door wetenschappelijke aannames, juist de werkelijkheid níet kon zien. En dat was in strijd met wat ik thuis hoorde van mijn vader. Die blies hoog van de toren over de kracht van wetenschap en de achterlijkheid van allerlei bijgeloof. Ik hoorde het woord ‘bijgeloof’ en dacht: blijkbaar is er ‘hoofdgeloof’ en ‘bijgeloof’. Hij deed het klinken alsof zijn hoofdgeloof een soort wetenschappelijke waarheid was, en de waarheid, dat was toch iets van de kerk? “Er kan maar één waarheid zijn,” verzekerde hij me, “de objectieve waarheid.” Maar dat zeiden de kerken toch ook allemaal? En ik had echt vrij duidelijk meegekregen dat de kerkelijke waarheid sowieso niet waar was. “Allemaal dom bijgeloof.”

Wist ik veel dat er een verschil was tussen wetenschapsfilosofie, weten\-schaps\-beöef\-ening, wetenschappelijke hypotheses, wetenschappelijke paradigma's, wetenschappelijke theorieën en min- of meer vastgestelde wetenschappelijke waarheden. Ik wist niet eens wat een hypothese was. Karl Popper? Thomas Kuhn? Nooit van gehoord.

Kortom: ik had geen idee waar ik het over had als zei dat ik “wetenschap ook maar een religie” vond. Maar ik vond het prachtig dat ik daarmee het bloed bij mijn vader en andere pompeuze heren onder de nagels vandaan kon krabben. Hoe irritant moet het geweest zijn dat die blaag zonder baard zelfs de simpelste, binnen hun vakgebied algemeen geaccepteerde, feiten niet wilde aannemen?

Een andere onfeilbare manier om mijn vader te irriteren was door het verkondigen van archaïsch religieuze overtuigingen. Evolutietheorie? Dat vond ik dikke onzin, omdat… Ik weet het niet meer eigenlijk; volgens mij kwam ik niet veel verder dan het argument van niet-reduceerbare complexiteit. De tegenstand kwam overigens nauwelijks verder dan een beroep op autoriteit. Dat en een soort samenzweerderig gefluister over de domme, bijgelovige meute die de waarheid niet aan zouden kunnen. De bedoeling zal zijn geweest dat ik niet loeiend in die meute had willen meelopen.

Zo klinkt het alsof atheïsme me met de paplepel werd ingegoten. Dat is een verkeerde indruk. Het waren losse flarden. Er was geen gericht onderricht. Bovendien was maar één van mijn ouders overtuigd atheïst. De ander – dochter van twee atheïstische biologen – geloofde in in die tijd in van allerlei New Age-achtigs: vorige levens, kristallen en dergelijke. Die losse geloofsflarden spraken mij meer aan dan het coherente – in mijn ogen: stramme – wereldbeeld van mijn vader. Nou ja, coherent… Meneer de wetenschapsfundamentalist geloofde tegelijk wel in zijn eigen anekdotische ervaringen met bijvoorbeeld homeopathie, teleportatie, en in Carlos Castaneda's volledige New Age oeuvre.

Ondanks dat titels zoals \emph{De Celestijnse Belofte} gretige aftrek vonden in ons huishouden en mijn vader meestal wel een boek over Oosterse-stijl verlichting van guru Andrew Cohen op zijn nachtkastje had liggen, is het niet heel onnauwkeurig om me een derde generatie atheïst te noemen, in ieder geval wat betreft de Abrahamitische god – die met de grote G. Op die éne ouderwetse god na, zijn er heden ter dage ontelbaar veel goden waar ik wel in geloof. Die goden zijn microben.

2 decennia terug, als puber, zat het me dwars dat de hooggeleerde heren wetenschappers om me heen meer wisten dan ik. Al dat gepronk met hun kennis kwam maar pompeus op me over. Het probleem voor mij was natuurlijk dat ik totaal niet met deze intellectuele zwaargewichten kon meekomen, en op mijn eigen eigenwijze manier leerde ik daarom een belangrijke wetenschaps-sceptische les: dat je kennis niet hoeft aan te nemen op basis van de autoriteit van de kennisdrager. Pas veel later leerde ik dat dit scepticisme één van de hoofdpijlers is van de wetenschap; er wordt nog altijd veel gebruik gemaakt van Newton's natuurkundige en wiskundige verworvenheden, maar er zijn maar weinig wetenschappers die Newton's alchemistische experimenten erg serieus nemen.

Ik als adoloscent zou Newton's alchemie juist erg gewaardeerd hebben, vooral vanwege de tegenstrijdigheid met de modern-wetenschappelijke status quo. Dat een ongetrainde knaap als ik met wat losse gedachtenflarden dieper tot de werkelijkheid door zou kunnen dringen dan het voltallig wetenschappelijke establishment vond ik geen buitensporige gedachte. Zo zaten mijn vader's wetenschapsanocdotes ook altijd in elkaar; als ik hem moest geloven, zat iedereen vanwege ego en eigenbelang vast in door hem (en andere innovatie avonturiers) al lang achterhaalde theoriën. Hele paradigma's moesten op de schop. Die houding erfde ik, maar de wetenschappelijke onderlegging liet nog even op zich wachten.

Voordat ik enige wetenschappelijke training mocht ontvangen, kwam ik voor het eerst ècht in aanraking met de evolutieleer, via \emph{The Selfish Gene} van Richard Dawkins. Ja, toen vond ik dat toch wel een mooier verhaal dan Genesis (waar ik me eerlijk gezegd nooit doorheen had weten te worstelen). Ontelbare BBC documentaires later was ik òm: wetenchap was toch wel een cool dingetje. (Vooral het driedelige \emph{The Cell} staat me nog bij.) Dus toen toch maar biologie studeren, net als mijn vader, mijn grootvader en oma.

Was ik dan voorgoed voorbij mijn irrationaliteit? Zou ik vanaf het binnengaan van die universiteit dan eindelijk de objectieve waarheid omarmen? Gelukkig werd dat geenszins van me gevraagd. Waar ik me de universiteit eerder als een soort kathedraal had voorgesteld, leek het in werkelijkheid meer op een soort dorp, of kleine stad. (Bijzonder aan de stad Groningen is dat de rijksuniversiteit feitelijk het zenuwcentrum van de stad is.) Wat steeds dieper tot me doordrong was dat het voor het beoefenen (en begrijpen) van wetenschap geen enkel probleem is een irritioneel dier te zijn, zolang je binnen je veld maar werkt volgens de wetenschappelijke methode, wat inhoud: toetsbare hypotheses stellen, hypotheses testen, en testresultaten delen. De toenemende omarming van de wetenschappelijke methode is juist arm in arm gegaan met de herkenning van hoe moeilijk het is om de de wereld objectief waar te nemen vanuit onze subjectieve werkelijkheid.

Wat me had tegengestaan aan wetenschap was helemaal niet de essentie van wetenschap: pomp. Het was niet meer dan een ongelukkig toeval dat de wetenschapper met de vaderbaard in mijn ouderlijk huis een nogal pompeus heerschap was. Hij deed het overkomen alsof hij als wetenschapper vóórbij de naïfiteit en het bijgeloof was. Zijn geloof was in de ware werkelijkheid. De schellen waren van zijn ogen gevallen. Een goede zelfpromotor en een slechte wetenschapspromotor. Het was één groot beroep op autoriteit. En zo werkt wetenschap niet, in principe.

In de praktijk zit er vaak niets anders op dan wetenschappelijke autoriteiten te vertrouwen op basis van die autoriteit. Idealiter vindt dit vertrouwen plaats op basis van het vertrouwen dat die collega's genieten onder collega's. Aanzien binnen de gangen van de wetenschapsinstituten is een goede graadmeter voor betrouwbaarheid. % TODO
Naar buiten toe werkt wetenschap wel zo. “Tegengestelde publicaties” “Dus ook maar meningen.”

%Microbiologie is voor micro-gelovigen de hoofdstam van de wetenschap. Vanuit deze stam vertakken de rest van de disciplines.

% de tijd dat mijn vader biologie studeerde en psychologisch onderzoek deed naar gedrag bij mensen en dieren. Mijn academische carrière zag er niet direct glansrijk uit.


\end{document}
