\documentclass[12pt,a4paper]{article}

\usepackage[a4paper]{geometry}
\usepackage[utf8]{inputenc}
\usepackage[unicode,final,colorlinks=true]{hyperref}        

\begin{document}

\title{Het eerste gebod}
\author{Rowan Rodrik van der Molen}
\date{juni 2018}
\maketitle


De vader van mijn beste vriend op de middelbare school genoot het nodige aanzien als hoogleraar sociale psychologie op de Rijksuniversiteit Groningen. Die vriend – Willem – en ik zaten samen op MAVO (tegenwoordig VMBO-TL) – niet bepaald het voorbereidend wetenschappelijk onderwijs. (Wij waren liever aan het PC-gamen dan aan het leren.) Mijn eigen ouders hadden allebei gymnasium gedaan. En toevallig bleken mijn vader en de vader van Willem elkaar al te kennen van de tijd dat mijn vader biologie studeerde en psychologisch onderzoek deed naar gedrag bij mensen en dieren. Zo vaders, zo zoons, behalve dat wij minder geïnteresseerd waren in wetenschappelijk onderzoek dan in elkaar katopschieten tijdens een potje Quake. Onze academische carrière zag er niet direct glansrijk uit.

Willem heeft zijn MAVO nog wel afgemaakt. Ik niet. Ik was met andere dingen bezig; nachtenlang gamen en over het wereldwijde web struinen bleken niet te combineren met 's ochtend opstaan en naar school fietsen. Na een tweede halfslachtige poging tot het voltooien van een eindexamenjaar werd ik vriendelijk verzocht om maar gewoon helemaal weg te blijven. En even later had ik een baan als programmeur. Zo ging dat rond de milleniumwissel. Willem kreeg trouwens een nog betere baan. De bubbel was zich nog aan het opblezen een bood genoeg ruimte voor jongens met een grote bek en weinig tot geen diploma's.

Op de MAVO werd weinig aandacht besteed aan wetenschappelijke vorming. De docenten waren in principe al blij als ze ons enigszins begrijpelijk konden leren schrijven en enigszins begrijpend konden leren lezen. En we moesten natuurlijk een paar simpele sommetjes leren. Op weg naar volwassenheid was mijn voornaamste aanraking met wetenschap via mijn persoonlijk omgeving en via de TV. Van \emph{The X-Files} leerde ik dat agent Mulder—vol fantastisch bijgeloof—het vaker bij het rechte eind had dan agent Scully met haar hoofd vol wetenschappelijk scepticisme. Het leek voor mij als jonge kijker alsof zij, verblind door wetenschapelijke aannames, juist de werkelijkheid níet kon zien. En dat was in strijd met wat ik thuis hoorde van mijn vader. Die blies hoog van de toren over de kracht van wetenschap en de achterlijkheid van allerlei bijgeloof. Ik hoorde het woord ‘bijgeloof’ en dacht: blijkbaar is er ‘hoofdgeloof’ en ‘bijgeloof’. Hij deed het klinken alsof zijn hoofdgeloof een soort wetenschappelijke waarheid was, en de waarheid, dat was toch iets van de kerk? “Er kan maar één waarheid zijn,” verzekerde hij me, “de objectieve waarheid.” Maar dat zeiden de kerken toch ook allemaal? En ik had echt vrij duidelijk meegekregen dat de kerkelijke waarheid sowieso niet waar was. “Allemaal dom bijgeloof.”

Wist ik veel dat er een verschil was tussen wetenschapsfilosofie, weten\-schaps\-beöef\-ening, wetenschappelijke hypotheses, wetenschappelijke paradigma's, wetenschappelijke theorieën en min- of meer vastgestelde wetenschappelijke waarheden. Ik wist niet eens wat een hypothese was. Karl Popper? Thomas Kuhn? Nooit van gehoord. Niemand die me de nuances van Newton's meer en minder wetenschappelijke werk had uitgelegd.

% XXX: tè verkrampte poging om mijn darlings in leven te houden? 
Kortom: ik had geen idee waar ik het over had als zei dat ik “wetenschap ook maar een religie” vond. Maar ik vond het prachtig dat ik daar het bloed bij mijn vader en de vader van Willem onder de nagels vandaan kon krabben. Hoe irritant moet het geweest zijn dat die blaag zonder baard zelfs de simpelste, binnen hun vakgebied algemeen geaccepteerde, feiten niet wilde aannemen? Want ik weigerde categorisch om hun op onderzoeksgegevens gebaseerde feiten te erkennen.

Atheïsme is me met de paplepel ingegoten. In ieder geval: het was duidelijk dat ‘we’ niet in (een) God geloofden. Mijn grootouders waren ook al niet kerkelijk. Van mijn moeder's kant hadden beide grootouders zelfs biologie gestudeerd. Mijn oma brak haar studie af om voorrang te geven aan haar biologische functie en mijn opa ging later aan de Rijksuniversiteit Groningen op zijn beurt doceren. Mijn vader kende zijn schoonvader al als (naar zijn zeggen nogal slaapverwekkende) docent voordat hij mijn moeder leerde kennen. Zijn eigen moeder—oma 2, die ik nooit gekend heb—was zendelinge geweest, en daardoor erg gedesillusioneerd geraakt met de kerk. Je kunt dus rustig zeggen dat ik een derde-generatie atheïst ben. Tenminste: ik ben een atheïst ten opzichte van de Christelijke god—die ene, met de hoofdletter G. Op die ouderwetse god na, zijn er ontelbaar veel goden waar ik wèl in geloof en die bovendien aantoonbaar bestaan.

De goden zijn microben. Dat is mijn antwoord op de vraag of er binnen de wetenschap ruimte is voor religie. En niet alleen is er binnen de wetenschap ruimte voor religie, wetenschap heeft religie nodig, en religie heeft wetenschap nodig. Er is tegenwoordig binnen beiden een schreinend gebrek aan beiden.

2 decennia terug, als puber, zat het me dwars dat de hooggeleerde heren wetenschappers om me heen meer wisten dan ik. Al dat gepronk met hun kennis kwam maar pompeus op me over. Het probleem voor mij was natuurlijk dat ik totaal niet met deze intellectuele zwaargewichten kon meekomen, en op mijn eigen eigenwijze manier leerde ik daarom een belangrijke wetenschaps-sceptische les: dat je kennis niet hoeft aan te nemen op basis van de authoriteit van de kennisdrager. Pas veel later leerde ik dat dit scepticisme één van de hoofdpijlers is van de wetenschap; er wordt nog altijd veel gebruik gemaakt van Newton's natuurkundige en wiskundige verworvenheden, maar er zijn maar weinig wetenschappers die Newton's alchemistische experimenten erg serieus nemen.

De adoloscente ik zou Newton's alchemie juist erg gewaardeerd hebben, vooral vanwege de tegenstrijdigheid met de modern-wetenschappelijke status quo. Dat een ongetrainde knaap als ik met wat losse gedachtenflarden dieper tot de werkelijkheid door zou kunnen dringen dan het voltallig wetenschappelijke establishment vond ik geen buitensporige gedachte. Zo zaten mijn vader's wetenschapsanocdotes ook altijd in elkaar; als ik hem moest geloven, zat iedereen vanwege ego en eigenbelang vast in door hem (en andere innovatie avonturiers) al lang achterhaalde theoriën. Hele paradigma's moesten op de schop. Die houding erfde ik, maar de wetenschappelijke onderlegging liet nog even op zich wachten.

Voordat ik enige training kreeg als wetenschapper, kwam ik voor het eerst ècht in aanraking met de evolutieleer, via \emph{The Selfish Gene} van Richard Dawkins. Ja, toen vond ik dat toch wel een mooier verhaal dan Genesis (waar ik me eerlijk gezegd nooit doorheen had weten te worstelen). Ontelbare BBC documentaires later was ik òm: wetenchap was toch wel een cool dingetje. (Vooral het driedelige \emph{The Cell} staat me nog bij.) Dus toen toch maar biologie studeren, net als mijn vader, mijn grootvader en oma.

Was ik dan voorgoed voorbij mijn irrationaliteit? Zou ik vanaf het binnengaan van die universiteit dan eindelijk de objectieve waarheid omarmen? Gelukkig werd dat geenszins van me gevraagd. Waar ik me de universiteit eerder als een soort kathedraal had voorgesteld, leek het in werkelijkheid meer op een soort dorp, of kleine stad. (Bijzonder aan de stad Groningen is dat de rijksuniversiteit feitelijk het zenuwcentrum van de stad is.) Wat steeds dieper tot me doordrong was dat het voor het beoefenen (en begrijpen) van wetenschap geen enkel probleem is een irritioneel dier te zijn, zolang je binnen je veld maar werkt volgens de wetenschappelijke methode, wat inhoud: toetsbare hypotheses stellen, hypotheses testen, en testresultaten delen. De toenemende omarming van de wetenschappelijke methode is juist arm in arm gegaan met de herkenning van hoe moeilijk het is om de de wereld objectief waar te nemen vanuit onze subjectieve werkelijkheid.

Wat me had tegengestaan aan wetenschap was helemaal niet de essentie van wetenschap. Ik verwachtte toch de nodige mate van pompeuziteit, maar 

% Die ontwikkelingen overspanden een kleine 2 decennia, gekenmerkt door een mythologisch moeras. Voortdurend werd ik heen en weer geslingerd tussen \emph{new-age}-achtige inzichten aan de ene kant, en conservatief-sceptische beschouwingen aan de andere kant. Tegelijkertijd moest ik leren navigeren tussen een overdreven geloof in authoriteit enerzijds en een totale overgave aan fantastisch samenzweringsdenken aan de andere kant. Eigenlijk was ik een wrak. De bron van mijn ellende was dat ik me had laten wijsmaken dat ik meer moest zijn dan dat; ik moest mijn potentieel vervolmaken, wat dat ook moge zijn. De herkenning van de schadelijkheid van die grootheidswaan leidde tot het eerste van zes geboden:

\paragraph{1. Gij zult klein zijn.} Het is niet dat ik een keus heb. Ik heb me altijd wel maximaal proberen op te blazen, maar die manie landde altijd in een depressie. Het heeft gewoon niet zoveel zin om me groter en geweldiger voor te doen dan ik ben. Van bovenaf gezien, ben ik best een klein diertje dat door slechts een handvol bacteriën tot pulp gereduceerd kan worden.

En daar zijn we bij het 

[Dit is moeilijk voor veel wetenschappers (en niet-wetenschappers).]


\end{document}
