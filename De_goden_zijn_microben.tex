\documentclass[12pt,a4paper]{article}

\usepackage[a4paper]{geometry}
\usepackage[utf8]{inputenc}
\usepackage{textcomp}
\usepackage[unicode,final,colorlinks=true]{hyperref}        

\begin{document}

\title{Het eerste gebod}
\author{Rowan Rodrik van der Molen}
\date{juni 2018}
\maketitle


De goden zijn microben – klein — zo klein dat ze onzichtbaar zijn. “Gij zult klein zijn,” is daarom een vanzelfsprekend eerste gebod voor micro-gelovigen. Een µ-gelovige word je vanzelf door door een microscoop te kijken en na te denken over wat je ziet.

De microscoop legt een wereld bloot die fundamenteler is dan de onze: de wereld van cellen. De meeste cellen zijn vrijlevend, maar sommige hebben zich in de loop van de evolutie in zulke complexe kolonies georganiseerd dat we die lichamen zien als organismen op zich. Als mens ben ik een gigantisch complex van cellen – allemaal met identiek DNA en per weefsel-type verschillende gen-expressie. Het is een duizelingwekkend idee: enkele tientallen biljoenen cellen die nauw genoeg kunnen samenwerken om een essay over religie en wetenschap te schrijven.

Microben – eencellige organismen – waren er al miljarden jaren voordat de eerste mens op het evolutionaire toneel verscheen. De mens is een tijdelijke vertoning in een soortenrijk waar meercelligen vooral bijrollen vervullen. Onze bijrol is bijzonder; we zijn er in geslaagd onze biosfeer om te vormen, maar zonder microben was er geen biosfeer, en nog lang nadat de resten van de laatste mens door microben zijn omgezet zal de biosfeer door microben gedomineerd worden.

Het menselijk beestje mag dan uit een indrukwekkend aantal cellen bestaan, al het dierlijk en plantaardig leven bij elkaar opgeteld stelt weinig voor tegenover de microbiële biomassa. Zelfs binnen ons eigen lichaam wordt het aantal microbiële cellen geschat op zo'n honderd biljoen, zo'n 3 tot 10 keer meer dan onze lichaamseigen cellen. In onze darmen zit ruim een kilo aan godengebroed.

Waarom noem ik die microben goden? Het lijkt misschien alsof ik een beetje grappig probeer te doen. Schijn bedriegt. Dit is een oprechte poging om je religieuze affiliatie te winnen. Microben bepalen in angstwekkende mate ons lot. Pas sinds de uitvinding van de microscoop in de zeventiende eeuw heeft de wetenschap schoorvoetend middelen ontwikkeld om bacteriën te kunnen bestrijden. (En nog steeds is necrotiserende fasciitis geen lolletje.) Inmiddels blijkt bovendien dat we die kilo microben in onze ingewanden juist nodig hebben. We kunnen de goden niet doden zonder onszelf te doden. Daarom willen wij – µ-gelovigen – ze beter begrijpen.

Om microben te begrijpen moet je je klein maken, heel klein – zo klein als ik me voelde als ongedisciplineerde MAVO-scholier in de laat-twintigste eeuw…  Mijn ouders hadden elkaar ontmoet op de universiteit, waar mijn vader biologie studeerde en mijn opa (van moeders kant) biologie doceerde. Ik kwam dus uit een intellectuele dynastie, en, zoals in zoveel intellectuele nesten, werd er dringend neergekeken op Jan met de Pet – de archetypische MAVO-klant.

De MAVO (tegenwoordig VMBO-TL) was niet bepaald het voorbereidend wetenschappelijk onderwijs. Op weg naar volwassenheid was mijn voornaamste aanraking met wetenschap via het hooggeleerde heerschap in mijn ouderlijk huis en via de TV. Van \emph{The X-Files} leerde ik dat agent Mulder—vol fantastisch bijgeloof—het vaker bij het rechte eind had dan agent Scully met haar hoofd vol wetenschappelijk scepticisme. Het leek voor mij als jonge kijker alsof zij, verblind door wetenschappelijke aannames, juist de werkelijkheid níet kon zien. En dat was in strijd met wat ik thuis hoorde van mijn vader; die blies hoog van de toren over de kracht van wetenschap en de achterlijkheid van allerlei bijgeloof. Ik hoorde het woord ‘bijgeloof’ en dacht: blijkbaar is er ‘hoofdgeloof’ en ‘bijgeloof’. Hij deed het klinken alsof zijn hoofdgeloof een soort wetenschappelijke waarheid was, en de waarheid, dat was toch iets van de kerk? “Er kan maar één waarheid zijn,” verzekerde hij me, “de objectieve waarheid.” Maar dat zeiden de kerken toch ook allemaal? En ik had echt vrij duidelijk meegekregen dat de kerkelijke waarheid sowieso niet waar was. “Allemaal dom bijgeloof.”

Wist ik veel dat er een verschil was tussen wetenschapsfilosofie, weten\-schaps\-beöef\-ening, wetenschappelijke hypotheses, wetenschappelijke paradigma's, wetenschappelijke theorieën en min- of meer vastgestelde wetenschappelijke waarheden. Ik wist niet eens wat een hypothese was. Karl Popper? Thomas Kuhn? Nooit van gehoord.

Kortom: ik had geen idee waar ik het over had wanneer ik zei dat ik “wetenschap ook maar een religie” vond. Maar ik vond het prachtig als ik daarmee het bloed bij mijn pompeuze vader onder de nagels vandaan kon krabben. Hoe irritant moet het geweest zijn dat die blaag zonder baard zelfs de simpelste, binnen hun vakgebied algemeen geaccepteerde, feiten niet wilde aannemen?

Een andere onfeilbare manier om mijn vader te irriteren was door het verkondigen van archaïsch religieuze overtuigingen. Evolutietheorie? Dat vond ik dikke onzin, omdat… Ik weet het niet meer eigenlijk; volgens mij kwam ik niet veel verder dan het argument van niet-reduceerbare complexiteit. De tegenstand kwam overigens nauwelijks verder dan een beroep op autoriteit. Dat en een soort samenzweerderig gefluister over de domme, bijgelovige meute die de waarheid niet aan zouden kunnen. De bedoeling zal zijn geweest dat ik niet loeiend in die meute had willen meelopen.

Toch werd atheïsme me niet met de paplepel werd ingegoten. Het waren losse flarden. Er was geen gericht onderricht. Bovendien was maar één van mijn ouders overtuigd atheïst. De ander geloofde in in die tijd in van allerlei New Age-achtigs: vorige levens, kristallen en dergelijke. Die losse geloofsflarden spraken mij meer aan dan het coherente – in mijn ogen: stramme – wereldbeeld van mijn vader. Meneer de wetenschapsfundamentalist geloofde tegelijk dan weer wel in zijn eigen anekdotische ervaringen met bijvoorbeeld homeopathie, teleportatie, en in Carlos Castaneda's volledige New Age oeuvre.

Ondanks dat titels zoals \emph{De Celestijnse Belofte} gretige aftrek vonden in ons huishouden en mijn vader meestal wel een boek over Oosterse-stijl verlichting van goeroe Andrew Cohen op zijn nachtkastje had liggen, is het niet heel onnauwkeurig om me een derde generatie atheïst te noemen, in ieder geval wat betreft de Abrahamitische god – die met de grote G. Op die éne ouderwetse god na, zijn er heden ter dage ontelbaar veel goden waar ik wel in geloof. Die goden zijn microben.

2 decennia terug, zat het me dwars dat de hooggeleerde heer wetenschapper, drs. P.P. v.d. Molen meer wist dan ik. Het probleem voor mij was natuurlijk dat ik totaal niet met dit intellectuele zwaargewicht kon meekomen, en op mijn eigen eigenwijze manier leerde ik daarom een belangrijke wetenschaps-sceptische les: dat je kennis niet hoeft aan te nemen op basis van de autoriteit van de kennisdrager. Pas veel later leerde ik dat dit scepticisme één van de hoofdpijlers is van de wetenschap; er wordt nog altijd veel gebruik gemaakt van Newton's natuurkundige en wiskundige verworvenheden, maar er zijn maar weinig wetenschappers die Newton's alchemistische experimenten erg serieus nemen.

Ik als adolescent zou Newton's alchemie juist erg gewaardeerd hebben, vooral vanwege de tegenstrijdigheid met de modern-wetenschappelijke status quo. Dat een ongetrainde knaap als ik met wat losse gedachtenflarden dieper tot de werke\-lijk\-heid door zou kunnen dringen dan het voltallig wetenschappelijke \emph{establishment} vond ik geen buitensporige gedachte. Zo zaten mijn vaders wetenschapsanecdotes ook altijd in elkaar; als ik hem moest geloven, zat iedereen vanwege ego en eigenbelang vast in door hem (en andere innovatie avonturiers) al lang achterhaalde theorieën. Hele paradigma's moesten op de schop. Die houding erfde ik, maar de wetenschappelijke onderbouwing liet nog even op zich wachten.

In de tussentijd, vulde ik mijn hoofd met veel meer van die oude, vertrouwde \emph{X-Files} sfeer: samenzweringenstheorieën, pseudowetenschap en New Age kolder. Bij alles ging ik er vanuit dat de consensus sowieso wel fout zou zijn. Bovendien: “voor elke publicatie is wel een tegenpublicatie te vinden.” Als ik terug kon reizen in de tijd, zou ik zuchtend tegen mezelf zeggen: heel knap dat je zo twijfelt, maar hier heb je een stapel boeken met de consensus in verschillende vakgebieden. Ja, met een dwarse geest kun je hele paradigma's omwerpen, maar je zult vaker ongelijk hebben dan gelijk. Gij zult klein zijn. (Ik zou de boeken niet hebben gelezen, want: “Geen tijd. Te druk met belangrijke dingen.”)

Voordat ik zelf enige wetenschappelijke training mocht ontvangen, kwam ik voor het eerst ècht in aanraking met de evolutieleer, via \emph{The Selfish Gene} van Richard Dawkins. Ja, toen vond ik dat toch wel een mooier verhaal dan Genesis (waar ik me eerlijk gezegd nooit doorheen had weten te worstelen). Ontelbare BBC documentaires later was ik òm: wetenschap was \emph{cool}. (Vooral het driedelige \emph{The Cell} staat me nog bij.) Er was alleen nog een paranoia trip op een stuk \emph{space cake} nodig om me in te laten zien dat ik toch meer wilde leren over de natuurlijke wereld. Na een toelatingsexamen – zwaar voor iemand die zijn MAVO nooit had afgemaakt – mocht ik biologie gaan studeren, net als voor mij mijn vader, mijn grootvader en oma.

Als tijdsgebonden organisme was er geen andere manier voor mij om meer te leren dan er tijd voor uit te trekken. Voordat ik deze kleinheid begon te accepteren, lukte het me niet om in te zien dat ik alleen meer over het leven van cellen kon leren door er tijd voor op te offeren. Deze opofferingsgezindheid vereiste een omarming van mijn kleinheid in de tijd.

Was ik dan voorgoed voorbij mijn irrationaliteit? Zou ik vanaf het binnengaan van die universiteit dan eindelijk de objectieve waarheid omarmen? Gelukkig werd dat geenszins van me gevraagd. Waar ik me de universiteit eerder als een soort kathedraal had voorgesteld, leek het in werkelijkheid meer op een soort dorp, of kleine stad. Wat steeds dieper tot me doordrong was dat het voor het beoefenen (en begrijpen) van wetenschap geen enkel probleem is een irrationeel dier te zijn, zolang je binnen je veld maar werkt volgens de wetenschappelijke methode, wat inhoud: toetsbare hypotheses stellen, hypotheses testen, en testresultaten delen. De toenemende omarming van de wetenschappelijke methode is juist arm in arm gegaan met de herkenning van hoe moeilijk het is om de de wereld objectief waar te nemen vanuit onze subjectieve werkelijkheid.

Wat me had tegengestaan aan wetenschap was helemaal niet de essentie van wetenschap: opgeblazen plechtigheid. Het was niet meer dan een ongelukkig toeval dat de wetenschapper met de vaderbaard in mijn ouderlijk huis een nogal pompeus heerschap was. Hij deed het overkomen alsof hij als wetenschapper vóórbij de naïviteit en het bijgeloof was. De schellen waren van zijn ogen gevallen. Een goede zelfpromotor en een slechte wetenschapspromotor. Het was één groot beroep op autoriteit. En zo werkt wetenschap niet, in principe.

“Gij zult klein zijn.” Hetzelfde besef van onze beperkingen dat de wetenschap groot heeft gemaakt zou een nieuw geloof in microben groot kunnen maken. We hebben ons moeten schikken in een perifiere plaats in het heelal, aan een zijtak aan de boom van levensvormen, in lichamen met zeer beperkte fysieke en intellectuele vermogens. Maar we aanbidden het grote: fictieve goden, superhelden, topatleten of andere supersuccesvollen. Terwijl het het kleine is wat het leven de moeite waard maakt. Sterker: het is het kleine dat het leven mogelijk maakt. We kunnen niet anders dan klein zijn. Al onze niet-fictieve helden weten dat: zij brengen dagelijkse offers om in een beperkt aantal dimensies uit te kunnen blinken. Intussen kan een enkele bacteriële infectie in een flits een einde maken aan hun glorie.

We weten dat microben bestaan. Ze vragen niet van ons dat we dagelijks op onze knieën vallen om te bidden, ondanks dat dat prima momenten zouden zijn om onze knieën binnen te dringen. Maar al die microben verdienen wel ontzag. Dankbaarheid aan de ene kant, bijvoorbeeld voor de cyanobacterieën (blauwalgen) die zuurstof in de atmosfeer hebben gebracht, of de gisten die water omzetten in wijn. Aan de andere kan hebben we iets nodig om te vervloeken als we door de willekeur op onze knieën worden gedwongen. Uiteindelijk is het allemaal op conto van die kleine krengen. Er is geen beter symbool om ons eraan te herinneren dat het leven offers eist.

\end{document}
