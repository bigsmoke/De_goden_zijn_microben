\documentclass[12pt,a4paper]{article}

\usepackage[a4paper]{geometry}
\usepackage[utf8]{inputenc}
\usepackage[unicode,final,colorlinks=true]{hyperref}        

\begin{document}

\title{De goden zijn microben}
\author{Rowan Rodrik van der Molen}
\date{juni 2018}
\maketitle


De vader van mijn beste vriend op de middelbare school genoot het nodige aanzien als hoogleraar sociale psychologie op de Rijksuniversiteit Groningen. Die vriend – Willem – en ik zaten samen op MAVO (tegenwoordig VMBO-TL) – niet bepaald het voorbereidend wetenschappelijk onderwijs. (Wij waren liever aan het PC-gamen dan aan het leren.) Mijn eigen ouders hadden allebei gymnasium gedaan. En toevallig bleken mijn vader en de vader van Willem elkaar al te kennen van de tijd dat mijn vader biologie studeerde en psychologisch onderzoek deed naar gedrag bij mensen en dieren. Zo vaders, zo zoons, behalve dat wij minder geïnteresseerd waren in wetenschappelijk onderzoek dan in elkaar katopschieten tijdens een potje Quake. Onze academische carrière zag er niet direct glansrijk uit.

Willem heeft zijn MAVO nog wel afgemaakt. Ik niet. Ik was met andere dingen bezig; nachtenlang gamen en over het wereldwijde web struinen bleken niet te combineren met 's ochtend opstaan en naar school fietsen. Na een tweede halfslachtige poging tot het voltooien van een eindexamenjaar werd ik vriendelijk verzocht om maar gewoon helemaal weg te blijven. En even later had ik een baan als programmeur. Zo ging dat rond de milleniumwissel. Willem kreeg trouwens een nog betere baan. De bubbel was zich nog aan het opblezen een bood genoeg ruimte voor jongens met een grote bek en weinig tot geen diploma's.

Op de MAVO werd weinig aandacht besteed aan wetenschappelijke vorming. De docenten waren in principe al blij als ze ons enigszins begrijpelijk konden leren schrijven en enigszins begrijpend konden leren lezen. En we moesten natuurlijk een paar simpele sommetjes leren. Op weg naar volwassenheid was mijn voornaamste aanraking met wetenschap via mijn persoonlijk omgeving en via de TV. Van \emph{The X-Files} leerde ik dat agent Mulder—vol fantastisch bijgeloof—het vaker bij het rechte eind had dan agent Scully met haar hoofd vol wetenschappelijk scepticisme. Het leek voor mij als jonge kijker alsof zij, verblind door wetenschapelijke aannames, juist de werkelijkheid níet kon zien. En dat was in strijd met wat ik thuis hoorde van mijn vader. Die blies hoog van de toren over de kracht van wetenschap en de achterlijkheid van allerlei bijgeloof. Ik hoorde het woord ‘bijgeloof’ en dacht: blijkbaar is er ‘hoofdgeloof’ en ‘bijgeloof’. Hij deed het klinken alsof zijn hoofdgeloof een soort wetenschappelijke waarheid was, en de waarheid, dat was toch iets van de kerk? “Er kan maar één waarheid zijn,” verzekerde hij me, “de objectieve waarheid.” Maar dat zeiden de kerken toch ook allemaal? En ik had echt vrij duidelijk meegekregen dat de kerkelijke waarheid sowieso niet waar was. “Allemaal dom bijgeloof.”

Wist ik veel dat er een verschil was tussen wetenschapsfilosofie, weten\-schaps\-beöef\-ening, wetenschappelijke hypotheses, wetenschappelijke paradigma's, wetenschappelijke theorieën en min- of meer vastgestelde wetenschappelijke waarheden. Ik wist niet eens wat een hypothese was. Karl Popper? Thomas Kuhn? Nooit van gehoord. Niemand die me de nuances van Newton's meer en minder wetenschappelijke werk had uitgelegd.

Kortom: ik had geen idee waar ik het over had als zei dat ik “wetenschap ook maar een religie” vond. Maar ik vond het prachtig dat ik daar het bloed bij mijn vader en de vader van Willem onder de nagels vandaan kon krabben. Hoe irritant moet het geweest zijn dat die blaag zonder baard zelfs de simpelste, binnen hun vakgebied algemeen geaccepteerde, feiten niet wilde aannemen? Want ik weigerde categorisch om hun op onderzoeksgegevens gebaseerde feiten te erkennen.

\par\noindent\rule{\textwidth}{0.4pt}


Mijn wetenschaps-scepticisme culmineerde zich in een creationistische fase en via het Internet flirtte ik met een paranoia soort Christenfundamentalisme.

Mijn grootouders van mijn moeder's kant hadden beiden biologie gestudeerd. Ook mijn vader was grotendeels atheïstisch opgevoed, en mijn ouders waren allebei zeer liberaal in hun politieke oriëntatie. Er was geen betere manier om me af te zetten dan door met bijbelse onzin aan 



Iedere wetenschapsfilosoof had me kunnen uitleggen dat de schoonheid van wetenschap juist is dat ik deze heren niet op hun blauwe ogen (of hun bombast) hoefde te geloven. Ongeacht hun en mijn persoonlijke gebreken, st

Maar de vraag is nu niet of wetenschap een religie is. De vraag is of er binnen de wetenschap ruimte is voor religie. Laat ik eerst eens te raad gaan bij mijn puberbrein. Mijn onvolgroeide, onzekere, overzelfverzekerde zelf zou moeiteloos het onderwerp gekaapt hebben met: “Dus wetenschap is een methode, geen religie; maar geloof in de wetenschap kan wel religieuze vormen aannemen.” En dan snap ik weer waarom men soms knarsetandend met mij aan tafel zat. Arme mannen. (Nou ja, ze waren best welvarend, zoals de meeste hoogopgeleide NRC lezers.)

Dawkins, een vokaal tegenstander van religie noemen het een “mislukte wetenschap”

Ik had niks met objectieve waarheid, want ik was niet erg geleerd en kon me als puber moeilijk omhoogvechten in de dominantiehierarchy op basis van mijn MAVO kennis. De waarheid was ook maar een mening, aangezien ik mijn mening tenminste kon verdedigen (op basis van virulente persoonlijk aanvallen).


Via een onweg kwam ik uiteindelijk toch in het wetenschappelijk onderwijs terecht.

De goden zijn microben.

Dat is mijn antwoord op de vraag of er binnen de wetenschap ruimte is voor religie. Ja. Niet alleen is er ruimte voor religie. Wetenschap heeft religie nodig. En religie heeft wetenschap nodig.

\end{document}
